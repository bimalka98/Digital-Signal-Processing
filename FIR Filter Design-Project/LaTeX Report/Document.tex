%  -----------------------------------------------------------------------------
%  Author         : Bimalka Piyaruwan Thalagala
%  GitHub         : https://github.com/bimalka98
%  Date Created   : 01.09.2020
%  Last Modified  : 15.02.2020
%  -----------------------------------------------------------------------------

\documentclass[a4paper,11pt]{article}%,twocolumn
\input{settings/packages}
\input{settings/page}
\input{settings/macros}
\usepackage[ framed, numbered]{matlab-prettifier}%framed,%
\usepackage{listings}

\begin{document}

\begin{titlepage}
\center % Center everything on the page

%-------------------------------------------------------------------------------------
%	HEADING SECTIONS
%------------------------------------------------------------------------------------
\textbf{\large Department of Electronic and Telecommunication Engineering}\\[0.5cm]
\textbf{\Large University of Moratuwa, Sri Lanka}\\[1cm]
\textbf{\large EN2073 - Analog and Digital Communications}\\[2cm]
\includegraphics[width=0.3\textwidth]{figures/uomlogo}\\[2cm]

	
%-------------------------------------------------------------------------------------
%	TITLE SECTION
%------------------------------------------------------------------------------------
\textbf{\Huge Assignment 01 } \\[5cm]


%----------------------------------------------------------------------------------------
%	MEMBERS SECTION
%----------------------------------------------------------------------------------------


\vfill

\textbf{\large Submitted by}\\[0.5cm]

{\large Thalagala B.P.}	\hspace{5mm} {\large 180631J }\\[1cm]

%----------------------------------------------------------------------------------------
%	DATE SECTION
%----------------------------------------------------------------------------------------

\textbf{\large Submitted on}\\[0.5cm]
\textbf{\Large \today} % Date, change the \today to a set date if you want to be precise

%----------------------------------------------------------------------------------------

\vfill % Fill the rest of the page with whitespace
\begin{center}
	
	\textbf{\textit{ \large * {\tt MATLAB }Code used to generate the plots are attached at the end.}}
\end{center}

\end{titlepage}
\tableofcontents

\begin{center}
	\textbf{\textit{* PDF is clickable}}
\end{center}



\textit{Note:}\\
\textit{Additionally all the materials related to Task can also be found at \url{}}



\pagebreak
%%-----------------------------------------------------------------------
\section{Modeling the RF propagation Using Matlab}
%%-----------------------------------------------------------------------
\subsection{Relationship between Free Space Path Loss and Frequency}

\textit{Consider  following meanings for the parameters}\\

\begin{tabular}{l l }
	$P_{RX}$ & = Received Power at the Receiving Antenna\\
	$P_{TX}$ & = Transmitted Power at the Transmitting Antenna\\
	$f$ & = Frequency of the wave in Hz\\
	$f_{GHz}$ & = Frequency of the wave in GHz\\
	$d$& = Distance between the antennas in m\\
	$d_{km}$& = Distance between the antennas in km\\
	$G_{TX}$& = Directive gain of the Transmitter\\
	$G_{RX}$& = Directive gain of the Receiver\\
	$c$& = Velocity of the electromagnetic waves in a vacuum\\

\end{tabular}\\[1cm]


\[
\begin{split}
10.\log_{10}(L) &= 10.\log_{10}(\frac{(4\pi.f.d)^2}{c^2})\\
L_{dB}& = 10.\log_{10}((4\pi.f.d)^2) - 10.\log_{10}(c^2)\\
&=20.\log_{10}(4\pi.f.d)-20.\log_{10}(c)\\
&=20.\log_{10}(4\pi)-20.\log_{10}(c) + 20.\log_{10}(f) + 20.\log_{10}(d)\\
&=20.\log_{10}(\frac{4\pi}{c}) + 20.\log_{10}(f) + 20.\log_{10}(d)\\
&= -147.5522168 + 20.\log_{10}(f_{GHz}.10^9) + 20.\log_{10}(d_{km}.10^3)\\
& = -147.5522168 + 20.\log_{10}(10^9)+ 20.\log_{10}(f_{GHz}) + 20.\log_{10}(10^3) + 20.\log_{10}(d_{km})\\
&= -147.5522168 + 180+ 20.\log_{10}(f_{GHz}) + 60 + 20.\log_{10}(d_{km})\\
&= -147.5522168 + 240+ 20.\log_{10}(f_{GHz}) + 20.\log_{10}(d_{km})\\
&= +92.44778322+20.\log_{10}(f_{GHz}) + 20.\log_{10}(d_{km})
\end{split}
\]

\textbf{\textit{Note : Axes of the following plots are given in the logarithmic scale and range of frequency was chosen from 50 GHz to 1000 GHz since some of the ITU-R models are only defined in the 10 GHz-1000 GHz range.}}

%\begin{figure}[!h]
%	\centering
%	\includegraphics[scale=0.35]{figures/FreeSpacePL.png}
%	\caption{Relationship between Free Space Path Loss and Frequency}
%\end{figure}



%%-----------------------------------------------------------------------
\pagebreak
\subsection{Rain attenuation, Fog attenuation and Atmospheric gas attenuation with Frequency}

\textit{Note : For the generation of following plots three of the Matlab built-in functions, namely \textbf{rainpl()\cite{matlab}, gaspl()\cite{matlab}, fogpl()\cite{matlab}} which are developed according to the ITU-R P Series recommendations were used and links for their documentations are given at the Reference section.}


\subsection{Variation of the Signal Power with the Distance}

\begin{tabular}{l r}
Chosen Carrier frequency & 50 GHz\\
Transmission power & 50 kW or 47 dB\\
Cable loss at Transmitter & 3 dB\\
Transmitter Gain & 30 dB\\
Receiver Gain &24.77 dB\\
Cable loss at Receiver &4 dB\\
Total Path Loss & Varies with Distance\\
\end{tabular}\\[1cm]

\subsection{Codes}


\lstinputlisting[basicstyle = \mlttfamily\scriptsize , style = Matlab-editor]{code/RFPL.m}


\pagebreak
\bibliographystyle{plain}
\bibliography{refer}

%---------------------------------------------------------------------------
\end{document}
